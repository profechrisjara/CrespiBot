% Options for packages loaded elsewhere
% Options for packages loaded elsewhere
\PassOptionsToPackage{unicode}{hyperref}
\PassOptionsToPackage{hyphens}{url}
\PassOptionsToPackage{dvipsnames,svgnames,x11names}{xcolor}
%
\documentclass[
  letterpaper,
  DIV=11,
  numbers=noendperiod]{scrreprt}
\usepackage{xcolor}
\usepackage{amsmath,amssymb}
\setcounter{secnumdepth}{5}
\usepackage{iftex}
\ifPDFTeX
  \usepackage[T1]{fontenc}
  \usepackage[utf8]{inputenc}
  \usepackage{textcomp} % provide euro and other symbols
\else % if luatex or xetex
  \usepackage{unicode-math} % this also loads fontspec
  \defaultfontfeatures{Scale=MatchLowercase}
  \defaultfontfeatures[\rmfamily]{Ligatures=TeX,Scale=1}
\fi
\usepackage{lmodern}
\ifPDFTeX\else
  % xetex/luatex font selection
\fi
% Use upquote if available, for straight quotes in verbatim environments
\IfFileExists{upquote.sty}{\usepackage{upquote}}{}
\IfFileExists{microtype.sty}{% use microtype if available
  \usepackage[]{microtype}
  \UseMicrotypeSet[protrusion]{basicmath} % disable protrusion for tt fonts
}{}
\makeatletter
\@ifundefined{KOMAClassName}{% if non-KOMA class
  \IfFileExists{parskip.sty}{%
    \usepackage{parskip}
  }{% else
    \setlength{\parindent}{0pt}
    \setlength{\parskip}{6pt plus 2pt minus 1pt}}
}{% if KOMA class
  \KOMAoptions{parskip=half}}
\makeatother
% Make \paragraph and \subparagraph free-standing
\makeatletter
\ifx\paragraph\undefined\else
  \let\oldparagraph\paragraph
  \renewcommand{\paragraph}{
    \@ifstar
      \xxxParagraphStar
      \xxxParagraphNoStar
  }
  \newcommand{\xxxParagraphStar}[1]{\oldparagraph*{#1}\mbox{}}
  \newcommand{\xxxParagraphNoStar}[1]{\oldparagraph{#1}\mbox{}}
\fi
\ifx\subparagraph\undefined\else
  \let\oldsubparagraph\subparagraph
  \renewcommand{\subparagraph}{
    \@ifstar
      \xxxSubParagraphStar
      \xxxSubParagraphNoStar
  }
  \newcommand{\xxxSubParagraphStar}[1]{\oldsubparagraph*{#1}\mbox{}}
  \newcommand{\xxxSubParagraphNoStar}[1]{\oldsubparagraph{#1}\mbox{}}
\fi
\makeatother


\usepackage{longtable,booktabs,array}
\usepackage{calc} % for calculating minipage widths
% Correct order of tables after \paragraph or \subparagraph
\usepackage{etoolbox}
\makeatletter
\patchcmd\longtable{\par}{\if@noskipsec\mbox{}\fi\par}{}{}
\makeatother
% Allow footnotes in longtable head/foot
\IfFileExists{footnotehyper.sty}{\usepackage{footnotehyper}}{\usepackage{footnote}}
\makesavenoteenv{longtable}
\usepackage{graphicx}
\makeatletter
\newsavebox\pandoc@box
\newcommand*\pandocbounded[1]{% scales image to fit in text height/width
  \sbox\pandoc@box{#1}%
  \Gscale@div\@tempa{\textheight}{\dimexpr\ht\pandoc@box+\dp\pandoc@box\relax}%
  \Gscale@div\@tempb{\linewidth}{\wd\pandoc@box}%
  \ifdim\@tempb\p@<\@tempa\p@\let\@tempa\@tempb\fi% select the smaller of both
  \ifdim\@tempa\p@<\p@\scalebox{\@tempa}{\usebox\pandoc@box}%
  \else\usebox{\pandoc@box}%
  \fi%
}
% Set default figure placement to htbp
\def\fps@figure{htbp}
\makeatother





\setlength{\emergencystretch}{3em} % prevent overfull lines

\providecommand{\tightlist}{%
  \setlength{\itemsep}{0pt}\setlength{\parskip}{0pt}}



 


\KOMAoption{captions}{tableheading}
\makeatletter
\@ifpackageloaded{bookmark}{}{\usepackage{bookmark}}
\makeatother
\makeatletter
\@ifpackageloaded{caption}{}{\usepackage{caption}}
\AtBeginDocument{%
\ifdefined\contentsname
  \renewcommand*\contentsname{Table of contents}
\else
  \newcommand\contentsname{Table of contents}
\fi
\ifdefined\listfigurename
  \renewcommand*\listfigurename{List of Figures}
\else
  \newcommand\listfigurename{List of Figures}
\fi
\ifdefined\listtablename
  \renewcommand*\listtablename{List of Tables}
\else
  \newcommand\listtablename{List of Tables}
\fi
\ifdefined\figurename
  \renewcommand*\figurename{Figure}
\else
  \newcommand\figurename{Figure}
\fi
\ifdefined\tablename
  \renewcommand*\tablename{Table}
\else
  \newcommand\tablename{Table}
\fi
}
\@ifpackageloaded{float}{}{\usepackage{float}}
\floatstyle{ruled}
\@ifundefined{c@chapter}{\newfloat{codelisting}{h}{lop}}{\newfloat{codelisting}{h}{lop}[chapter]}
\floatname{codelisting}{Listing}
\newcommand*\listoflistings{\listof{codelisting}{List of Listings}}
\makeatother
\makeatletter
\makeatother
\makeatletter
\@ifpackageloaded{caption}{}{\usepackage{caption}}
\@ifpackageloaded{subcaption}{}{\usepackage{subcaption}}
\makeatother
\usepackage{bookmark}
\IfFileExists{xurl.sty}{\usepackage{xurl}}{} % add URL line breaks if available
\urlstyle{same}
\hypersetup{
  pdftitle={CrespiBot},
  pdfauthor={Tec. Sup. Christopher Jara},
  colorlinks=true,
  linkcolor={blue},
  filecolor={Maroon},
  citecolor={Blue},
  urlcolor={Blue},
  pdfcreator={LaTeX via pandoc}}


\title{CrespiBot}
\author{Tec. Sup. Christopher Jara}
\date{2025-04-10}
\begin{document}
\maketitle

\renewcommand*\contentsname{Table of contents}
{
\hypersetup{linkcolor=}
\setcounter{tocdepth}{2}
\tableofcontents
}

\bookmarksetup{startatroot}

\chapter*{Compromiso y Objetivo}\label{compromiso-y-objetivo}
\addcontentsline{toc}{chapter}{Compromiso y Objetivo}

\markboth{Compromiso y Objetivo}{Compromiso y Objetivo}

\section*{Queridos estudiantes,}\label{queridos-estudiantes}
\addcontentsline{toc}{section}{Queridos estudiantes,}

\markright{Queridos estudiantes,}

\begin{quote}
El Concurso CrespiBot es una oportunidad única para demostrar sus
habilidades y creatividad en el campo de la robótica. Participar en este
evento no solo les permitirá representar a nuestra institución en
concursos nacionales, sino también desarrollar competencias que serán
esenciales para su futuro. La robótica educativa es una puerta hacia la
innovación y el conocimiento, y su participación en estos eventos les
abrirá nuevas oportunidades y experiencias.

¡Comprométanse con el Concurso CrespiBot, el circuito de Robótica
Educativa Regional y los concursos nacionales de robótica! Su esfuerzo y
dedicación serán recompensados, y juntos, podemos alcanzar grandes
logros y contribuir a la transformación social.
\end{quote}

¡Adelante, CrespiBots! ¡El futuro está en sus manos!

\section*{Beneficios del deporte ``Robótica
Educativa''}\label{beneficios-del-deporte-robuxf3tica-educativa}
\addcontentsline{toc}{section}{Beneficios del deporte ``Robótica
Educativa''}

\markright{Beneficios del deporte ``Robótica Educativa''}

La robótica educativa desempeña un papel fundamental en la formación
académica y personal de los estudiantes de la Unidad Educativa Carlos
Crespi II. A continuación, se detallan algunos de los beneficios más
significativos:

\textbf{Desarrollo de Habilidades Técnicas:} La robótica permite a los
estudiantes adquirir conocimientos prácticos en áreas como la
informática, electricidad y mecánica. Estos conocimientos son esenciales
para su futuro profesional en un mundo cada vez más tecnológico.

\textbf{Fomento de la Creatividad y la Innovación:} Al diseñar y
construir robots, los estudiantes desarrollan su capacidad creativa y su
habilidad para resolver problemas de manera innovadora. Esto les prepara
para enfrentar desafíos complejos en su vida académica y profesional.

\textbf{Aprendizaje Interdisciplinario:} La robótica integra diversas
disciplinas, lo que facilita un aprendizaje más completo y enriquecedor.
Los estudiantes aprenden a aplicar conceptos de matemáticas, física y
programación en proyectos reales, lo que fortalece su comprensión y
habilidades.

\textbf{Desarrollo Personal:} La robótica educativa también contribuye
al desarrollo personal de los estudiantes, fomentando valores como la
perseverancia, el trabajo en equipo y la responsabilidad. Estos valores
son fundamentales para su formación integral y su crecimiento como
individuos.

\section*{Modalidades}\label{modalidades}
\addcontentsline{toc}{section}{Modalidades}

\markright{Modalidades}

Por estas y muchas mas razones, la ``Unidad Educativa Particular Carlos
Crespi II'', lo invita a participar en alguna de las modalidades que
este 2025 tendremos en nuestro evento:

\begin{enumerate}
\def\labelenumi{\arabic{enumi}.}
\item
  Carrera de Insectos Amateur.
\item
  Robots Seguidores de Linea Amateur.
\item
  Robots de Sumo Amateur 10 x 10
\item
  Campeonato de Soccer Amateur 15 x 15
\end{enumerate}

Y para poder disfrutar del evento lo invito a leer el reglamento oficial
para este campeonato, Bienvenido!\\
\strut \\
Link de Inscripción

\url{https://tinyurl.com/InscripcionesCrespiBot}

\bookmarksetup{startatroot}

\chapter{Reglamento del campeonato de soccer
amateur}\label{reglamento-del-campeonato-de-soccer-amateur}

\section{DESCRIPCIÓN GENERAL}\label{descripciuxf3n-general}

En esta categoría, se enfrentan dos equipos en una cancha de fútbol.
Cada equipo consta de dos (2) robots que se desplazarán guiados por dos
operadores a través de controles de comunicación inalámbrica. El partido
de Robot Soccer consta de dos tiempos en los cuales los robots
conducirán la pelota para anotar un gol en la cancha del equipo
contrario. Una vez culminado el segundo tiempo, el jurado determinará al
equipo ganador según el desarrollo de los equipos durante el partido.

\section{INDICACIONES GENERALES:}\label{indicaciones-generales}

En esta categoría se considerará como edad máxima 18 años. La única
persona que puede dirigirse al juez será el capitán del equipo. No se
podrá utilizar materiales externos que ensucien la pista, ya que en
algunos casos favorece a un equipo, pero puede perjudicar a los demás.
En caso de empate, se decidirá la victoria por penales. En caso de dos
faltas, se otorgará un penal al equipo contrario. La retención
voluntaria del balón utilizada para pasar tiempo se considera falta. Si
el mecanismo de pateo ocasiona cualquier tipo de daño (revisado por el
juez), ese equipo automáticamente perderá el partido. Se considera gol
cuando el balón cruza el 100\% de la línea de meta (arco); puede ser por
arrastre o disparo del mecanismo de pateo.

\section{DIMENSIONES Y CARACTERÍSTICAS DE LOS
ROBOTS}\label{dimensiones-y-caracteruxedsticas-de-los-robots}

Deberán tener un máximo de 15x15 centímetros (largo y ancho). No hay
restricción con respecto a la altura. La masa del robot es de máximo 500
gr. Incluyendo todos sus componentes.

\section{DISEÑO:}\label{diseuxf1o}

El diseño de cada robot es libre. El tipo de tracción del robot es
libre: ruedas, banda tipo oruga, patas articuladas, tipo gusano, etc. El
orificio para manejo de pelota debe ser máximo de 3 cm. Se permitirá el
uso de un mecanismo de pateo siempre y cuando no supere las medidas
máximas en estado activo, y no ocasione retención del balón.

\section{MATERIALES:}\label{materiales}

El empleo de materiales en la estructura del robot y el número de
motores es libre. El robot debe construirse con motores con caja de
reducción amarilla o azul, sin modificaciones internas o externas. La
alimentación eléctrica del robot será con pilas o baterías; está
prohibido el uso de combustibles (motores de combustión) o cualquier
material inflamable. Ningún robot deberá alimentarse en forma externa a
través de cables.

\section{CONTROL:}\label{control}

Se utilizará cualquier tipo de comunicación inalámbrica.

\section{BALÓN}\label{baluxf3n}

Deberá ser una pelota de golf estándar: 46 g de peso aproximadamente. 43
mm de diámetro aproximadamente. Si el balón se vuelve defectuoso durante
el encuentro, se procederá a detener el encuentro para reemplazar el
balón y se continuará en la misma posición en la que se detuvo el
encuentro. El balón se reemplazará únicamente con la autorización del
árbitro.

\section{CANCHA}\label{cancha}

Las dimensiones de la cancha se publicarán en la página oficial del
evento.

\section{LA COMPETENCIA. ROL DE
ENCUENTROS}\label{la-competencia.-rol-de-encuentros}

Se determinarán los turnos del partido por medio de un sorteo, el mismo
que se realizará en presencia de los jueces del evento y deberá ser
acatado a cabalidad. Los equipos se enfrentarán en un partido de dos (2)
tiempos de dos (2) minutos cada uno, con un tiempo intermedio de un (1)
minuto.

\section{RUTINA DE CADA PARTIDO}\label{rutina-de-cada-partido}

Se designará al equipo que empieza el juego por medio de un sorteo
realizado por el juez previo al encuentro. Los operadores de los robots
participantes entrarán a la zona de juego y ubicarán a sus respectivos
robots en la posición de inicio según corresponda. Los operadores del
robot se situarán detrás del arco correspondiente a su equipo.

\section{INTERRUPCIÓN Y REANUDACIÓN DEL
PARTIDO}\label{interrupciuxf3n-y-reanudaciuxf3n-del-partido}

El partido se interrumpirá a petición de los jueces cuando: Si uno o
ambos robots presentan defectos, el operador podrá solicitar tiempo (1
minuto y solo 1 vez por cada tiempo) para repararlo. Se podrá realizar
cambios de batería al finalizar cada encuentro, considerando que no
afecte dimensiones ni peso. Cuando el partido haya sido interrumpido, se
volverá a empezar desde la posición inicial. Los jueces podrán detener
el partido siempre que lo consideren necesario para deliberar o permitir
la entrada de los operadores a la zona del partido. La última decisión
siempre la tendrán los jueces y será inapelable. Cuando alguno de los
robots haya sufrido algún daño provocado por el equipo contrario o por
otras causas, que le impida continuar participando.

\section{FALTAS}\label{faltas}

Se consideran faltas las siguientes acciones, que serán sancionadas por
los jueces: Tardar más de 1 minuto en reanudar el partido después de
haber solicitado tiempo. Acelerar o mover el robot antes de que el
árbitro lo indique. Empujar al robot del equipo contrario cuando el
balón esté a una distancia mayor a 10 cm del afectado.

\section{DESCALIFICACIÓN DEL
ENCUENTRO}\label{descalificaciuxf3n-del-encuentro}

Se considerarán como descalificación (perdiendo automáticamente el
partido) las siguientes acciones: Causar desperfectos de forma
deliberada al oponente. Agredir verbalmente al juez o a los miembros del
equipo oponente. El uso de dispositivos inflamables.

\section{EL INCUMPLIMIENTO DE CUALQUIERA DE LAS DISPOSICIONES DEL
REGLAMENTO.}\label{el-incumplimiento-de-cualquiera-de-las-disposiciones-del-reglamento.}

DESCALIFICACIÓN DE LA COMPETENCIA En casos extremos, los jueces se
reservan el derecho a expulsar de la competición a quienes se crean
merecedores de dicha sanción.

\section{JUECES}\label{jueces}

La figura del juez o los jueces es importante en la competencia; él será
el encargado de que las reglas y normas establecidas sean cumplidas. Los
jueces para esta competencia serán designados por el comité organizador.
Los participantes pueden presentar sus objeciones al juez encargado de
la categoría antes de que acabe la competencia. En caso de duda en la
aplicación de las normas, el juez se regirá por el reglamento
establecido haciéndolo cumplir.

\section{TRANSITORIOS:}\label{transitorios}

De no contar con un mínimo de 4 robots o equipos participantes, la
categoría será considerada únicamente como exhibición. Todos aquellos
sucesos que no se contemplen dentro del presente reglamento, durante la
competencia, serán resueltos por el Comité Organizador en conjunto con
los jueces, sin derecho de apelación. Una vez realizada la inscripción
del robot, no se realizarán devoluciones de dinero. Los distintivos de
cada participante serán entregados durante el desarrollo del evento. El
cronograma oficial se dará a conocer unos días antes de la competencia,
el cual podrá variar en base a los imprevistos.

\bookmarksetup{startatroot}

\chapter{Reglamento del campeonato de sumo
amateur}\label{reglamento-del-campeonato-de-sumo-amateur}

De la misma manera que en las artes marciales japonesas tradicionales,
los robots intentan empujar al competidor fuera del dohyō.

\section{MUY IMPORTANTE}\label{muy-importante}

El uso de activadores para todas las categorías autónomas es
obligatorio; el modelo de receptor a utilizar será el de RobotChallenge
Rumania o similares, como es de la marca JA-BOTS, esto con el fin de
evitar interferencias al momento de realizar la activación de los
robots.

\begin{center}
\pandocbounded{\includegraphics[keepaspectratio]{images/módulo_arranque_sumo.webp}}
\end{center}

\section{DEFINICIÓN DEL COMBATE DE
SUMO}\label{definiciuxf3n-del-combate-de-sumo}

Equipo conformado por 2 integrantes. Un partido se pelea entre dos
equipos; cada equipo tiene uno o más contendientes. Únicamente 2
miembros del equipo pueden encontrarse en el área de combate cumpliendo
función de Operador - Asistente, siendo el Operador el único que
controle al robot. Otros miembros del equipo (apoyo) deben mirar desde
la audiencia. De acuerdo con las reglas del juego (en lo sucesivo,
``estas reglas''), cada equipo compite en un Dohyo (Dohyo de sumo) con
un robot que ellos mismos han adquirido, modificado o construido según
las especificaciones de los ``Requisitos para los robots''. El partido
comienza a la orden del juez y continúa hasta que un concursante gana
dos puntos Yuko. El juez determina al ganador del partido.

\section{REQUISITOS PARA LOS ROBOTS DE LA CATEGORÍA
AMATEUR}\label{requisitos-para-los-robots-de-la-categoruxeda-amateur}

\subsection{Especificaciones generales del
robot}\label{especificaciones-generales-del-robot}

Las siguientes son especificaciones para todos los robots. El robot debe
construirse con motores con caja amarilla o azul plástica sin
modificaciones internas o externas; se acepta el uso de pesos metálicos,
más no elementos que tengan filo que puedan atentar contra la seguridad
de los participantes (cuchillas o navajas). La medida del robot para
sumo en la categoría amateur es de 10 cm x 10 cm, sin un límite de
altura, y su peso no debe ser superior a los 500 gr. El robot debe caber
dentro de un cubo cuadrado de las dimensiones apropiadas para cada
clase, asumiendo el 1\% de tolerancia de la medida. El robot puede
expandirse en tamaño después de que comience un encuentro, pero no debe
separarse físicamente en pedazos y debe permanecer como un solo robot
centralizado. Los pies del robot no deben ampliarse durante el
encuentro. Los robots que violen estas restricciones serán
descalificados. Los tornillos, tuercas y otras partes del robot con una
masa total de menos de 5 gr. que se caigan del cuerpo de un robot no
deben causar la pérdida del round. Los robots que usan navajas o
cuchillas finas las cuales puedan romperse durante un encuentro, solo en
el caso de que la cuchilla se rompe durante el encuentro, el
participante dispondrá de 3 min para cambiar o retirar la cuchilla; si
la cuchilla se encuentra rota al inicio del round, el robot pierde
inmediatamente ese encuentro. Todos los robots deben ser autónomos;
puede emplearse cualquier mecanismo de control, siempre que todos los
componentes estén contenidos dentro del robot y el mecanismo no
interactúe con un sistema de control externo (humano, máquina u otro).
El robot obtiene un número para fines de registro. Muestre este número
en su robot para permitir que los espectadores y oficiales identifiquen
su robot. Este número se colocará donde el participante considere
pertinente.

\section{RESTRICCIONES DE ROBOTS}\label{restricciones-de-robots}

Dispositivos de interferencia, como luces estroboscópicas, luz tipo
flash, etc. Destinados a saturar los sensores IR de los oponentes, no
están permitidos. No se permiten piezas que puedan romper o dañar el
Dohyo. No utilice piezas que están destinadas a dañar el robot del
oponente o su operador. Los empujones y golpes normales no se consideran
intención de dañar. Dispositivos que pueden almacenar líquido, polvo,
gas u otras sustancias para arrojar al oponente no están permitidos. No
se permiten dispositivos con llamas. No se permiten dispositivos que
arrojen cosas a tu oponente. No se permiten sustancias pegajosas para
mejorar la tracción. Los neumáticos y otros componentes del robot en
contacto con el Dohyo no deben poder levantar y sostener un papel A4
estándar (80 g/m2) durante más de cinco segundos. Los dispositivos para
aumentar la fuerza descendente, como bombas de vacío e imanes, no están
permitidos. Todos los bordes, incluidos, entre otros, la pala frontal,
no deben estar lo suficientemente afilados como para rayar o dañar el
ring, otros robots o jugadores.

\section{REQUISITOS PARA EL DOHYO
(SUMO)}\label{requisitos-para-el-dohyo-sumo}

\subsection{Interior de Dohyo}\label{interior-de-dohyo}

El interior del Dohyo se define como la superficie de juego rodeada e
incluyendo la línea fronteriza. Cualquier lugar fuera de esta área se
llama exterior del Dohyo. \#\#\# Especificaciones de Dohyo El Dohyo
deberá ser de forma circular y de las dimensiones exactas. La línea del
borde está marcada como un Dohyo circular blanco mate de un ancho
apropiado para la clase dada en el borde exterior de la superficie de
juego. El área del Dohyo se extiende hasta el borde exterior de esta
línea circular. Para todas las dimensiones dadas del Dohyo se aplica una
tolerancia del 5\%. Líneas de inicio (Shikiri-Sen) color café tendrán
una longitud de 100 mm o, en su defecto, se puede utilizar otro elemento
que simule las mismas que se deben retirar antes de iniciar el partido.
La superficie del Dohyo debe ser lisa, pintada o de vinil de colores
negro y blanco mate.

\begin{center}
\pandocbounded{\includegraphics[keepaspectratio]{images/doyo_sumo_amateur.webp}}
\end{center}

\subsection{Exterior de Dohyo}\label{exterior-de-dohyo}

Debe haber un espacio apropiado para cada clase de sumo fuera del borde
exterior del ring. Este espacio puede ser de cualquier color excepto
blanco ni retroreflectivo, y debe de ser material blando de cualquier
forma siempre y cuando no se violen los conceptos básicos de estas
reglas. Esta área, con el Dohyo en el medio, se denominará ``área del
Dohyo''. Cualquier marca o parte de la plataforma del ring fuera de las
dimensiones mínimas también se considerará en el área del ring. \#\#
DINÁMICA DEL PARTIDO DE SUMO Un partido consta de 3 rondas, cada una de
2 minutos máximo. El equipo que gana dos rondas o recibe dos puntos
``Yuko'' primero, dentro del límite de tiempo, deberá ganar el partido.
Un equipo recibe un punto ``Yuko'' cuando gana una ronda. Si alcanza el
límite de tiempo antes de que un equipo pueda obtener dos puntos
``Yuko'', y uno de los equipos ha recibido un punto Yuko, el equipo con
un punto Yuko ganará. Cuando en las 3 rondas no se ha ganado por ninguno
de los equipos dentro del límite de tiempo, se podrá disputar una ronda
adicional, durante la cual ganará el equipo que reciba el primer punto
Yuko. Alternativamente, el ganador/perdedor del partido puede ser
decidido por los jueces, por sorteo o por revancha. Se otorgará un punto
Yuko al ganador cuando se solicite la decisión de los jueces o se
empleen sorteos.

\subsection{CURSO DE LA COMPETICIÓN}\label{curso-de-la-competiciuxf3n}

¡¡¡Importante!!! Una persona puede ser operador de un máximo de 2
robots.

Los robots se dividirán en grupos según el número de participantes. La
competencia se llevará a cabo en un sistema de
grupos/cuartos/semifinales/finales para permitir tantas rondas de juego
para cada robot. Cada partido se juega al mejor de 3 rondas y será
supervisado por 2 árbitros (uno principal y un asistente), excepto las
finales, donde el partido se disputa en el sistema de eliminación
directa. Si dos robots de un mismo equipo avanzan a
cuartos/semifinales/finales y jugarán uno contra el otro, deberán jugar
el partido, sin exigir que uno de ellos avance sin jugar, ni exigir que
se arreglen los partidos o los opositores. El orden de los robots en los
grupos será aleatorio, se hará después de la apertura oficial de la
competencia y estará disponible en el sitio web para todos los
participantes. Los que superen los grupos jugarán
cuartos/semifinales/final. Si el número de participantes no fuera
suficiente para los grupos, la competición se jugará desde el principio
utilizando el sistema todos contra todos. La posición será aleatoria. Un
equipo tiene derecho a 2 interrupciones de reprogramación, de 2 minutos
cada una. Esta regla se aplica solo durante un partido en curso. Además
de los partidos, se permiten los cambios y reprogramaciones. Los equipos
deben estar en el inicio en un máximo de 1 minuto desde la solicitud; de
lo contrario, perderán el partido. No existirá mesa de retención de
robots; todos los equipos permanecerán en la sala reservada para ellos
(la sala estará marcada en el mapa como área de pits). Los equipos
pueden salir de la sala solo cuando son llamados al área de competencia.
Cada equipo será llamado por un oficial de competencia cuando necesite
ir a la sala de espera que se encuentra cerca del área de competencia.
Solo existirá mesa de homologación. Después de la homologación, los
equipos que seguirán en la salida permanecerán en el área de
competición, en la zona de espera. Los equipos saldrán de esta área solo
si el árbitro está de acuerdo, o solo para reparaciones, y deben
regresar en el tiempo establecido por el árbitro. Si el equipo no
regresa a la primera llamada, perderá el partido. Terminado un partido,
los equipos deben regresar a la sala reservada para ellos. Cada equipo
tiene la responsabilidad de seguir la parrilla de salida (horario), que
se muestra en el sitio web y en el espacio del equipo.

\section{INICIAR, DETENER, REANUDAR, FINALIZAR UNA
PARTIDA}\label{iniciar-detener-reanudar-finalizar-una-partida}

Colocación de robots Siguiendo las instrucciones del juez, los dos
equipos se acercan al ring para colocar sus robots en el ring. Los
operadores colocarán los robots al mismo tiempo en el ring. El juez dará
la señal. Después de la colocación, los robots ya no se pueden mover.
Cualquier parte de los robots debe colocarse detrás del Shikiri-Sen
(líneas de observación). El robot no cruzará la línea de salida hacia el
oponente. El robot debe colocarse sobre y dentro de las líneas
extendidas verticalmente desde ambos bordes de Shikiri-Sen (línea de
inicio).

\section{COMIENZO DE LA CONTIENDA}\label{comienzo-de-la-contienda}

El juez anuncia el comienzo de la ronda. Los equipos comienzan sus
robots; no existe tiempo de seguridad, los robots pueden empezar a
funcionar inmediatamente. Después de la señal de encendido, los
jugadores deben salir del área del ring.

\section{DETENER, REANUDAR}\label{detener-reanudar}

El partido se detiene y se reanuda cuando un juez lo anuncia.

\section{FIN}\label{fin}

El partido termina cuando el juez lo anuncia. Los dos equipos recuperan
los robots del área del ring.

\section{PUNTUACIÓN}\label{puntuaciuxf3n}

Puntuación para robots: se otorgará un punto Yuko cuando: Un equipo
obliga legalmente al cuerpo del robot contrario a tocar el espacio
exterior. el Dohyo, que incluye el lado lateral del Dohyo en sí mismo.
El robot oponente ha tocado el espacio fuera del ring por sí solo.
Cualquiera de las anteriores tiene lugar al mismo tiempo que finaliza el
partido anunciado. Cuando un robot con ruedas ha volado y ha caído sobre
el ring o en condiciones similares, Yuko no se contará y el partido
continúa.

\section{DECISIÓN POR CRITERIO DE LOS
JUECES}\label{decisiuxf3n-por-criterio-de-los-jueces}

Cuando se requiera la decisión de los jueces para decidir el ganador, se
tendrán en cuenta los siguientes puntos, tomados en consideración:
Méritos técnicos en el movimiento y funcionamiento de un robot. Puntos
de penalización durante el partido Actitud de los jugadores durante el
partido.

\section{REVANCHA}\label{revancha}

Los robots están enredados u orbitando entre sí sin progreso perceptible
durante 5 segundos. Si no está claro si se está progresando o no, el
juez puede extender el límite de tiempo para el progreso observable
hasta 30 segundos. Ambos robots se mueven, sin avanzar, o se detienen
(exactamente al mismo tiempo) y permanecen parados durante 5 segundos
sin tocarse. Sin embargo, si un robot detiene su movimiento primero,
después de 5 segundos se declarará que no tiene voluntad para luchar. En
este caso, el oponente recibirá un Yuko, incluso si el oponente también
se detiene. Si ambos robots se están moviendo y no está claro si se está
progresando o no, el juez puede extender el límite de tiempo hasta 30
segundos. Si ambos robots tocan el exterior del Dohyo aproximadamente al
mismo tiempo, y no puede determinarse quién tocó primero, se convoca una
revancha. \#\# SOLICITUD PARA DETENER EL PARTIDO Un jugador puede
solicitar detener el juego cuando está lesionado o su robot tuvo un
accidente y el juego no puede continuar. \#\# NO SE PUEDE CONTINUAR EL
PARTIDO Cuando el juego no puede continuar debido a una lesión del
jugador o un accidente del robot, el jugador que es la causa de dicha
lesión o accidente pierde el partido. Cuando no esté claro qué equipo es
la causa, el equipo que no puede continuar el juego, o que solicita
detener el juego, será declarado perdedor.

\#\# DETERMINACIÓN DEL GANADOR El robot con más puntos será determinado
como el ganador del partido. En caso de empate en la puntuación final,
el juez votará por el ganador según la táctica, la agresividad y la
actividad. Si ninguno de los robots anotó un punto, el juez puede
decidir que no hay ganador del partido.

\section{DECLARACIÓN DE OBJECIONES}\label{declaraciuxf3n-de-objeciones}

No se declararán objeciones contra las decisiones de los jueces. La
persona líder de un equipo puede presentar objeciones al Comité, antes
de que finalice el partido, si existen dudas en el ejercicio de estas
reglas. Si no hay miembros del Comité presentes, la objeción se puede
presentar al juez. antes de que termine el partido.

\section{JUECES}\label{jueces-1}

La figura del juez o los jueces es importante en la competencia; él será
el encargado de que las reglas y normas establecidas sean cumplidas. Los
jueces para esta competencia serán designados por el comité organizador.
Los participantes pueden presentar sus objeciones al juez encargado de
la categoría antes de que acabe la competencia. En caso de duda en la
aplicación de las normas, el juez se regirá por el reglamento
establecido haciéndolo cumplir.

\section{TRANSITORIOS:}\label{transitorios-1}

De no contar con un mínimo de 4 robots o equipos participantes, la
categoría será considerada únicamente como exhibición. Todos aquellos
sucesos que no se contemplen dentro del presente reglamento, durante la
competencia, serán resueltos por el Comité Organizador en conjunto con
los jueces, sin derecho de apelación. Una vez realizada la inscripción
del robot, no se realizarán devoluciones de dinero. Los distintivos de
cada participante serán entregados durante el desarrollo del evento. El
cronograma oficial se dará a conocer unos días antes de la competencia,
el cual podrá variar en base a los imprevistos.

\bookmarksetup{startatroot}

\chapter*{References}\label{references}
\addcontentsline{toc}{chapter}{References}

\markboth{References}{References}

\phantomsection\label{refs}




\end{document}
